%%%%%%%%%%%%%%%%%%%%%%%%%%%%%%%%%%%%%%%%%%%%%%%%%%%%%%%%%%%%%%%%%%%%%%%%%%%%%%%%
%2345678901234567890123456789012345678901234567890123456789012345678901234567890
%        1         2         3         4         5         6         7         8

\documentclass[letterpaper, 10 pt, conference]{ieeeconf}  % Comment this line out if you need a4paper

%\documentclass[a4paper, 10pt, conference]{ieeeconf}      % Use this line for a4 paper

\IEEEoverridecommandlockouts                              % This command is only needed if 
                                                          % you want to use the \thanks command

\overrideIEEEmargins                                      % Needed to meet printer requirements.

%In case you encounter the following error:
%Error 1010 The PDF file may be corrupt (unable to open PDF file) OR
%Error 1000 An error occurred while parsing a contents stream. Unable to analyze the PDF file.
%This is a known problem with pdfLaTeX conversion filter. The file cannot be opened with acrobat reader
%Please use one of the alternatives below to circumvent this error by uncommenting one or the other
%\pdfobjcompresslevel=0
%\pdfminorversion=4

% See the \addtolength command later in the file to balance the column lengths
% on the last page of the document

% The following packages can be found on http:\\www.ctan.org
%\usepackage{graphics} % for pdf, bitmapped graphics files
%\usepackage{epsfig} % for postscript graphics files
%\usepackage{mathptmx} % assumes new font selection scheme installed
%\usepackage{times} % assumes new font selection scheme installed
%\usepackage{amsmath} % assumes amsmath package installed
%\usepackage{amssymb}  % assumes amsmath package installed

% \usepackage{algorithm}
\usepackage{algorithmic}
\usepackage{amsmath}
\usepackage{relsize}
\usepackage{amsfonts}
\usepackage{amssymb}
\usepackage{mathtools}
\usepackage{graphicx}
\usepackage{epsfig}
\usepackage{color}
\usepackage{mathtools}
\usepackage{tikz}
\usepackage{relsize}
\usepackage{float}
\usepackage{dsfont}
\usepackage{hyperref}
\usepackage[nameinlink]{cleveref}
\newcommand{\bigqm}[1][1]{\text{\larger[#1]{\textbf{?}}}}
\usepackage{tikz,fullpage}
\usetikzlibrary{arrows, petri, topaths}
\usepackage{tkz-berge}
\usepackage[position=top]{subfig}
\usepackage{verbatim}
\usepackage{amsthm}
\usepackage{pgf}
\usepackage{tikz}
\usepackage{fancyhdr, setspace, color, soul}
\usepackage{geometry,polyglossia,fontspec,csquotes, doi}
\usepackage{ucs}
\usepackage[utf8x]{inputenc}
% \usepackage[english,hebrew]{babel}
\usepackage{breqn}
\usepackage{mdframed}
\usepackage{dsfont}
\usepackage{tabularx}
\usepackage{xcolor}
\usepackage{xspace}
\usepackage{cjhebrew}
\usepackage{caption}
\usepackage{subcaption}
\usepackage{subfig}
\usepackage{ascii}
\usepackage{multirow}
\usepackage{xfrac}
\usetikzlibrary{arrows, automata, backgrounds,snakes}
\newfontfamily\hebrewfont{Times New Roman}[Script=Hebrew]

% define MACROS
\newcommand{\len}{15}
\newcommand{\LPart}{0.4}
\DeclarePairedDelimiter{\ceil}{\lceil}{\rceil}
\DeclarePairedDelimiter{\floor}{\lfloor}{\rfloor}
\newtheorem{theorem}{Theorem}[section]
\newtheorem{corollary}{Corollary}[theorem]
\newtheorem{lemma}[theorem]{Lemma}
\newtheorem*{remark}{Remark}
\newcounter{casenum}
\newenvironment{caseof}{\setcounter{casenum}{1}}{\vskip.5\baselineskip}
\newcommand{\case}[2]{\vskip.5\baselineskip\par\noindent {\bfseries Case \arabic{casenum}:} #1\\#2\addtocounter{casenum}{1}}
\newcommand\rob{\ensuremath{r}\xspace}
\newcommand\opp{\ensuremath{o}\xspace}
\newcommand{\w}{\ensuremath{W}\xspace}
\newcommand{\fcc}{\ensuremath{FCC}\xspace}
\newcommand{\gipc}{\ensuremath{GIPC}\xspace}
\newcommand{\cros}{\ensuremath{CROS}\xspace}
\newcommand{\coos}{\ensuremath{COS}\xspace}
\newcommand{\gn}{\ensuremath{GN}\xspace}
\newcommand{\gf}{\ensuremath{GF}\xspace}
\newcommand{\go}{\ensuremath{GO}\xspace}
\DeclarePairedDelimiter\abs{\lvert}{\rvert}%
\DeclareMathOperator*{\argmax}{arg\,max} % Jan Hlavacek
\allowdisplaybreaks[2]
\newtheorem{definition}{Definition}
\def\checkmark{\tikz\fill[scale=0.4](0,.35) -- (.25,0) -- (1,.7) -- (.25,.15) -- cycle;}
\def\uncheckmark{$\mathbin{\tikz [x=1.4ex,y=1.4ex,line width=.2ex] \draw (0,0) -- (1,1) (0,1) -- (1,0);}$}
\newcommand{\Cross}{$\mathbin{\tikz [x=1.4ex,y=1.4ex,line width=.2ex, red] \draw (0,0) -- (1,1) (0,1) -- (1,0);}$}%
\usepackage{algorithm}
\usepackage{algorithmic}
\usepackage{amsmath,amssymb,amsthm}
\usepackage{breqn}
\usepackage{seqsplit}
\usepackage{relsize}
\usepackage{amsfonts}
\usepackage{mathtools}
\usepackage{graphicx}
\usepackage{epsfig}
\usepackage{color}
\usepackage{mathtools}
\usepackage{tikz}
\usepackage{relsize}
\usepackage{float}
\usepackage{dsfont}
\usepackage{hyperref}
\usepackage[nameinlink]{cleveref}
\newcommand{\bigqm}[1][1]{\text{\larger[#1]{\textbf{?}}}}
\usepackage{tikz,fullpage}
\usetikzlibrary{arrows, petri, topaths}
\usepackage{tkz-berge}
% \usepackage[position=top]{subfig}
\usepackage{verbatim}
\usepackage{pgf}
\usepackage{tikz}
\usepackage{fancyhdr, setspace, color, soul}
\usepackage{geometry,polyglossia,fontspec,csquotes, doi}
\usepackage{ucs}
\usepackage{breqn}
\usepackage{mdframed}
\usepackage{dsfont}
\usepackage{tabularx}
\usepackage{xcolor}
\usepackage{xspace}
\usepackage{caption}
\usepackage{multirow}
\usepackage{xfrac}
\usepackage{subcaption}
\usepackage{subfig}
\usetikzlibrary{arrows, automata, backgrounds,snakes}

% define MACROS
\newcommand{\len}{15}
\newcommand{\LPart}{0.4}
\DeclarePairedDelimiter{\ceil}{\lceil}{\rceil}
\DeclarePairedDelimiter{\floor}{\lfloor}{\rfloor}
\newtheorem{theorem}{Theorem}[section]
\newtheorem{corollary}{Corollary}[theorem]
\newtheorem{lemma}[theorem]{Lemma}
\newtheorem*{remark}{Remark}
\newcounter{casenum}
\newenvironment{caseof}{\setcounter{casenum}{1}}{\vskip.5\baselineskip}
\newcommand{\case}[2]{\vskip.5\baselineskip\par\noindent {\bfseries Case \arabic{casenum}:} #1\\#2\addtocounter{casenum}{1}}
\newcommand\rob{\ensuremath{r}\xspace}
\newcommand\opp{\ensuremath{o}\xspace}
\newcommand{\w}{\ensuremath{W}\xspace}
\newcommand{\fcc}{\ensuremath{FCC}\xspace}
\newcommand{\gipc}{\ensuremath{GIPC}\xspace}
\newcommand{\cros}{\ensuremath{CROS}\xspace}
\newcommand{\coos}{\ensuremath{COS}\xspace}
\newcommand{\gn}{\ensuremath{GN}\xspace}
\newcommand{\gf}{\ensuremath{GF}\xspace}
\newcommand{\go}{\ensuremath{GO}\xspace}
\DeclarePairedDelimiter\abs{\lvert}{\rvert}%
\DeclareMathOperator*{\argmax}{arg\,max} % Jan Hlavacek
\allowdisplaybreaks
\newtheorem{definition}{Definition}
\def\checkmark{\tikz\fill[scale=0.4](0,.35) -- (.25,0) -- (1,.7) -- (.25,.15) -- cycle;}
\def\uncheckmark{$\mathbin{\tikz [x=1.4ex,y=1.4ex,line width=.2ex] \draw (0,0) -- (1,1) (0,1) -- (1,0);}$}
\newcommand{\Cross}{$\mathbin{\tikz [x=1.4ex,y=1.4ex,line width=.2ex, red] \draw (0,0) -- (1,1) (0,1) -- (1,0);}$}%

% \renewcommand{\qedsymbol}{$\blacksquare$}

\title{\LARGE \bf
Competitive  Coverage:  (Full)  Information  as  a  Game  Changer
}


\author{Moshe N. Samson$^{1}$ and Dr. Noa Agmon$^{2}$% <-this % stops a space
\thanks{*This work was not supported by any organization}% <-this % stops a space
\thanks{$^{1}$Moshe N. Samson is with the Computer Science Department,
        Bar-Ilan University, Ramat Gan 5290002, Israel
        {\tt\small samson.moshe@gmail.com}}%
\thanks{$^{2}$Noa Agmon is with the Computer Science Department, Bar-Ilan University, Ramat Gan 5290002, Israel
        {\tt\small agmon@cs.biu.ac.il}}%
}


\begin{document}

\maketitle
\thispagestyle{empty}
\pagestyle{empty}


%%%%%%%%%%%%%%%%%%%%%%%%%%%%%%%%%%%%%%%%%%%%%%%%%%%%%%%%%%%%%%%%%%%%%%%%%%%%%%%%
\begin{abstract}

This paper introduces a new problem in robotic coverage, in which a robot \rob competes with another robot \opp in order to be the first to cover an area. In the variant discussed in this paper, the {\em asymmetric competitive coverage}, \opp is unaware of the existence of \rob, which attempts to take that fact into consideration in order to succeed in being the first to cover as many parts of the environment as possible. We consider different information models of \rob, that define how much it knows about the location of \opp and its planned coverage path. We show that unless \rob has full knowledge about \opp's location and coverage path, the behavior of \rob will remain the same. However, we show that when only the initial location of \opp is known to \rob, there is a correlation between the time it takes \rob to cover \opp's initial location and the quality of the result.

\end{abstract}


%%%%%%%%%%%%%%%%%%%%%%%%%%%%%%%%%%%%%%%%%%%%%%%%%%%%%%%%%%%%%%%%%%%%%%%%%%%%%%%%
\section{INTRODUCTION}

The robotic coverage problem is one of the fundamental problems in robotic research, and as such has received considerable attention in the past two decades \cite{galceran2013survey}. The problem has its theoretical interest, but is of special interest due to its immediate applicability in real world settings, such as cleaning, coating, demining and search and rescue. 

In the original problem of robotic coverage, a robot's goal is to determine a path that will visit each point in a given area at least once, usually while minimizing the time for completion \cite{galceran2013survey}. The problem has been examined in different settings, for example offline vs. online coverage \cite{gabriely2001spanning,agmon2008giving}, where the environment map is given in advance or discovered during the execution (respectively), coverage with the presence of threats that might stop the robot \cite{yehoshua2014safest}, and continuous vs. discrete domains \cite{gabriely2001spanning,yang2004neural}. In the multi-robot coverage problem, the coverage is a collaborative effort: each point in the area should be visited at least once by some robot from the team, and the common goal is to minimize the maximal working time of some robot from the team. 

In this work we formally define a new variant of the coverage problem, {\em competitive coverage}, in which robots do not work collaboratively, but competitively. More formally, two robots, \rob and \opp, are to cover a given area represented as a grid, and our goal is to maximize the number of cells \rob covers first, before they are covered by \opp.

The problem can be classified as {\em symmetric} or {\em asymmetric}, which refers to wither \opp even knows about \rob's existence or not, regardless to what is the extent of such information. We first examine the {\em asymmetric} variant of the problem, in which \opp operates without the knowledge of \rob's existence, and \rob knows it should compete with \opp. The problem is also modeled by the level of information one robot has on the other (beside its existence): initial location and strategy/coverage-path. We consider four different models: (i) \rob knows the initial location of \opp and its planned coverage path ; (ii) \rob knows only the path, but does not know \opp's initial location ; (iii) \rob knows \opp's initial location, but not its coverage path ; (iv) \rob does not know both \opp's path and its location.

Solving the competitive coverage problem, in some cases, is as computationally hard as solving the original coverage problem \cite{arkin2000approximation}. For the sake of the analysis, we consider environments in which an optimal coverage path can be computed in polynomial time, for example by using the Spanning-Tree Coverage Algorithm \cite{gabriely2001spanning}, which generate, cyclic coverage paths under some assumptions on the environment. 

We therefore present an optimal algorithm for \opp in the full-information case, and show that, surprisingly, having some information is equivalent (in the average case) to having no information at all. That is, if \rob has information only on \opp's location {\em or} on its path, its optimal behavior is exactly as if it does not know any of those. We support our theoretical analysis with computer simulation, demonstrating our findings. 

\section{Background and Related Work}
\input{SharedParts/related_work.tex}

\section{Competitive Coverage: Definition}
Let \rob and \opp be two robots, operating in an obstacle-free grid \w of size $N=m \times n$. Both robots move in the four basic directions (North, South, East, West). Consider \rob to be our robot-of-interest, and \opp to be the opponent. 
The goal of each robot is to cover the area, that is, find a path (denoted as the coverage path) that visits  each point in the area at least once. We define a coverage strategy of a robot as the coverage path, including the order of cells visited (specifically in a cyclic coverage path, the strategy indicates both the cells' ordering, and the direction of movement---clockwise or counterclockwise), or a behavior (for example: opponent chooses randomly, at each step, one of its four neighbors, and go there). We denote \rob's and \opp's strategy by $S_\rob,S_\opp\in \mathcal{S}$ (respectively), where $\cal{S}$ stands for the possible strategies space. I the offline version of the competitive coverage problem, $S_\rob$ and $S_\opp$ are computed in advance (before the execution), and mostly consists of cells permutation, where in the online problem, the strategies are behaviors, able to adjust online to the environment.

\opp is covering \w using an optimal coverage strategy, that is, it follows a path guaranteeing coverage in minimal time (we use in our experiment the Spanning-Tree Coverage (STC) algorithm \cite{gabriely2001spanning}). \rob's goal is to cover as many cells as possible from \w {\em before they are visited by \rob}. The calculation of the covering strategy of \rob, $S_\rob$, is based on its initial location, $i_r$. The initial position of \opp, $i_\opp$, is not necessarily known to \rob.

The way we define the problem, \rob can be given $i_\opp$, $S_\opp$, both or neither. These types of information are called {\em Information Models}, and defined as follows:
\begin{definition}[\textbf{Information Model}]
Information Model $IM \in \lbrace \varnothing, \lbrace S,I\rbrace \rbrace$, represents the knowledge a robot has on its opponent, where $S\in \mathcal{S} \cup \lbrace S_\emptyset \rbrace$, where $S_\emptyset$ stands for an unknown strategy, and $I \in W \cup \lbrace \w_\emptyset \rbrace$, where $\w_{\emptyset}$ refers to an unknown initial point. 
If $IM=\varnothing$ then the player of interest does not know its opponent exists.
Let $IM_\rob$ be the information model \rob is given about \opp, and let $IM_\opp$ be the information model \opp is given about \rob.
\end{definition}

\opp can operate with or without the knowledge of \rob's existence. We refer to these cases as the {\em symmetric and asymmetric competitive coverage}.
In the {\em asymmetric} case, $IM_\opp=\varnothing$, and in the {\em symmetric} case, $IM_\opp \neq \varnothing$. In both cases, $IM_\rob \neq \varnothing $.

\rob thrives first to maximize the number of cells it covers before \opp. Denote this value by 
\begin{dmath*}[compact]
\fcc_{x}(S_\rob, S_\opp, i_\rob, i_\opp )= 
\# \lbrace \textnormal{ First Covered Cells by } x \in \lbrace\rob,\opp\rbrace \rbrace 
\end{dmath*}
When deciding between options with the same \fcc value, \rob will choose the one that yields the fastest world's coverage time - the time it takes \rob to covers all of \w.

Considering all we said above, we define the competitive coverage problem:
\begin{mdframed}[backgroundcolor=gray!20] 
\begin{definition}[\textbf{Competitive Coverage Problem}]
Let \w be a finite, obstacles-free grid of size $N$. Let $IM_\rob=\lbrace S_\opp ,i_\opp \rbrace$ the information model given to \rob, and $IM_\opp \in \lbrace \varnothing, \lbrace S_\rob, i_\rob \rbrace \rbrace$. Find:
\begin{dmath*}[compact]
S_\rob^\star \in \mathcal{S} \textnormal{ s.t. } S_\rob^\star =\argmax_{S_\rob\in\mathcal{S}} \lbrace \fcc_{r}(S_\rob, S_\opp, i_\rob, i_\opp ) \rbrace
\end{dmath*}
\end{definition}
\end{mdframed}


\section{2D World - Results}
Same as before, we start by considering different information models: either \rob is given, or not, the initial location of \opp, $i_\opp$, and its strategy, $S_\opp$. Then, assuming a very basic behavior, where only \rob is aware of \opp's existence, and it must decide its strategy before the game begins, we tried to find the best strategy $S_\rob$ for \rob, given the information model.
Given this information model, we start by considering the asymmetric case in which only \rob is aware of \opp. Hence, we aim at determining the optimal strategy $S_\rob$ for \rob, based on the information models.

As mentioned above in the problem definition, each player holds an information model, that can contains the opponents initial position, strategy, neither or both. Considering that, we divide our research into 4 different information models:
\begin{enumerate}
\item \textbf{Full information} - $i_\opp$ is known, $S_\opp$ is known
\item \textbf{Zero-Information} - $i_\opp$ is known, $S_\opp$ is unknown
\item \textbf{Partial Information} -  $i_\opp$ is unknown, $S_\opp$ is known
\item \textbf{Partial Information} - $i_\opp$ is unknown, $S_\opp$ is unknown
\end{enumerate}

For each of these models, we would like to know what is the best strategy for \rob to play, maximizing $\fcc_\rob$.

\subsection{Full Information - asymmetrical-Knowledge}

In the full information case we present a new algorithm, \gipc (\ref{algorithms: gipc}) that maximizes the \fcc target function.
\begin{algorithm}
\begin{algorithmic}
	\STATE $i_p^{S_\rob S_\opp} \leftarrow $ Find Interception Point between $S_\rob$ and $S_\opp$
    \STATE GoTo $i_p^{S_\rob S_\opp}$, precede \opp by one step
    \LOOP
    	\IF {Cell $c_i$ \NOT Covered already}
        	\STATE GoTo $S_\opp(c_i)$
        \ENDIF
    \ENDLOOP
  
\end{algorithmic}
\caption{GIPC\label{algorithms: gipc}}
\end{algorithm}

\begin{theorem}
\gipc is correct, complete and optimal, and on a rectangular grid of size $m\times n$ it yields 
\begin{dmath*}[compact]
\mathbb{E}[\fcc] = m\cdot n-\frac{\left(m+1\right)\left(m-1\right)}{3m}-\frac{\left(n+1\right)\left(n-1\right)}{3n}
\end{dmath*}
\end{theorem}

% \input{SharedParts/GIPC_Correctness_progress.tex}

Correctness and completeness proofs of \gipc are out of the scope of this paper, and we will only prove the expected \fcc of \gipc.
In order to do that, we prove lemma \ref{lemmas:ExpectedDistanceTwoCellsRectangular}.

\begin{lemma}
The expected distance between two cells on a rectangular grid of size $m\times n$ is \[\frac{\left(m+1\right)\left(m-1\right)}{3m}+\frac{\left(n+1\right)\left(n-1\right)}{3n}\] 
\label{lemmas:ExpectedDistanceTwoCellsRectangular}
\end{lemma}
\begin{proof}
Let $X_1,Y_1,X_2,Y_2$ be random variables, indicating the coordinates for cell $C_1=\lbrace X_1, Y_1 \rbrace$ and  cell $C_2=\lbrace X_2, Y_2 \rbrace$. $X_1,X_2$ can fall anywhere in the range $\left[1\ldots m\right]$, where $Y_1,Y_2$ can fall anywhere in the range $\left[1\ldots n\right]$.
The expected distance between two cell is:
\begin{dmath*}[compact]
\mathbb{E}\left[\abs*{C_2-C_1}\right]=\mathbb{E}\left[\abs*{X_2-X_1} + \abs*{Y_2-Y_1}\right]=\mathbb{E}\left[\abs*{X_2-X_1}\right]+\mathbb{E}\left[\abs*{Y_2-Y_1}\right]\end{dmath*}

We can see that:
\begin{dmath*}[compact]
\mathbb{E}\left[\abs*{X_2-X_1}\right]=\ldots =\frac{\left(m+1\right)\left(m-1\right)}{3m}
\end{dmath*}
Where the full mathematical proof is in figure \ref{figures: proof expected fcc gipc}Note that the range changes from $m$ to $n$ when computing entity for $Y_1,Y_2$ instead of $X_1, X_2$.

\end{proof}

\begin{figure*}[thb]
      \centering
      \begin{multline*}
\mathbb{E}\left[\abs*{X_1-X_2}\right] =
\sum_{x_1=1}^{m}\sum_{x_2=1}^{m}{\frac{\abs*{x_1-x_2}}{m^2}} = \sum_{x_1=1}^{m}\sum_{x_2=1}^{x_1}{\frac{x_1-x_2}{m^2}}+\sum_{x_1=1}^{m}\sum_{x_2=x_1+1}^{m}{\frac{x_2-x_1}{m^2}}=\\
\frac{1}{m^2}\left(\sum_{x_1=1}^{m}\sum_{x_2=1}^{x_1}{x_1-x_2}+\sum_{x_1=1}^{m}\sum_{x_2=x_1+1}^{m}{x_2-x_1}\right)=\\
\frac{1}{m^2}\left(\sum_{x_1=1}^{m}{\left(x_1^2-\sum_{x_2=1}^{m}{x_2}\right)}\right)+
\frac{1}{m^2}\left(\sum_{x_1=1}^{m}\left(\sum_{x_2=x_1+1}^{m}{y}-\left(m-x_1\right)\cdot x_1\right)\right)=\\
\frac{1}{m^2}\left(\sum_{x_1=1}^{m}{\left(x_1^2-\frac{1}{2}\cdot x_1\left(x_1+1\right)\right)}\right)+
\frac{1}{m^2}\sum_{x_1=1}^{m}\left(\frac{1}{2}\cdot \left(m-x_1\right)\left(m+x_1+1)-\left(m-x_1\right)\cdot x_1\right)\right)=\\
\frac{1}{m^2}\sum_{x_1=1}^{m}{\left(x_1^2-\left(1+m\right)x_1+\left(\frac{1}{2}m^2+\frac{1}{2}m\right)\right)}=
\frac{m\left(\frac{1}{2}m^2+\frac{1}{2}m\right)}{m^2}+\frac{1}{m^2}\sum_{x_1=1}^{m}{\left(x_1^2-\left(1+m\right)x_1\right)}=\\
\frac{m+1}{2}+\frac{1}{m^2}\sum_{x_1=1}^{m}{x_1^2}-\frac{1}{m^2}\left(1+m\right)\left(\frac{1}{2}m\left(m+1\right)\right)=\\
\frac{m+1}{2}=\frac{1}{m^2}\left(\frac{1}{6}\cdot m\left(m+1\right)\left(2m+1\right)\right)-
\frac{1}{m^2}\left(1+m\right)\left(\frac{1}{2}m\left(m+1\right)\right)=\\
\frac{1}{6m}\left(3m^2+3m+2m^2+3m+1-3m^2-6m-3\right)=\frac{\left(m+1\right)\left(m-1\right)}{3m}
\end{multline*}
      %\includegraphics[scale=1.0]{figurefile}
\caption{Computing expected distance between x coordinates of two random cells}
\label{figures: proof expected fcc gipc}
 \end{figure*}

Using Lemma \ref{lemmas:ExpectedDistanceTwoCellsRectangular}, we infer the value $\mathbb{E}[\fcc]$ over world of size $m\times n$, using GIPC as $S_{\rob}$, to be:
\begin{dmath*}[compact]
\mathbb{E}\left[\fcc\right]=m\cdot n-\frac{\left(m+1\right)\left(m-1\right)}{3m}-\frac{\left(n+1\right)\left(n-1\right)}{3n}
\end{dmath*}

\subsection{Zero Information - asymmetrical-Knowledge}
In the zero-information case, \rob knows neither $i_\opp$ nor $S_\opp$. In fact, in this information model, \rob knows about \opp only that it exists.

Let us introduce the \cros algorithm (\ref{algorithms: cros}). We presents theoretical proof (Theorem \ref{theorems: cros correctness and optimalty}) to its optimality, and deduce that, in fact, the knowledge that an opponents exists in the world, does not grant \rob any advantage. Lastly, we prove the resulted $\mathbb{E}[\fcc]$ in Theorem \ref{theorems: cros correctness, optimalty and fcc}.

\begin{algorithm}
\begin{algorithmic}
	\STATE Choose $S_\rob \in \mathcal{S}$ in random.
    \STATE $c_i \leftarrow i_\rob$
    \LOOP
    	\STATE $c_i \leftarrow S_\rob(c_i)$
    \ENDLOOP
  
\end{algorithmic}
\caption{\cros\label{algorithms: cros}}
\end{algorithm}

\begin{theorem} \label{theorems: cros correctness, optimalty and fcc}
Given $IM_\rob=\lbrace S_\emptyset, i_\emptyset \rbrace$, \cros is correct, complete and when and optimal, and on a rectangular grid of size $m\times n$ yields  $\mathbb{E}[\fcc_\rob(S_\rob=\cros)]=\frac{N+1}{2}$.
\end{theorem}
% \begin{proof}

% \end{proof}

% \begin{theorem} \label{theorems: cross efcc}
% Given $IM_\rob=\lbrace S_\emptyset, i_\emptyset \rbrace$, then $\mathbb{E}[\fcc_\rob(S_\rob=\cros)]=\frac{N+1}{2}$.
% \end{theorem}
\begin{proof}
Same as before, the correctness and completeness proofs of \cros are out of scope of this context, and we would prove only the expected \fcc of using \cros.
First notice the following: for cell $c_i$ at time $i$, we say that $c_i$ is 'gained' by \rob \textbf{iff} its covering-time (by $S_o$) value (the time it was first covered by robot \opp) - ${CT}_{\opp}(c_i)$ - is higher than the time it was covered by robot $\rob$ - ${CT}_{\rob}(c_i)$.
Now, we can re-write the expression for \rob's gain to use property of covering time:
\begin{dmath*}
\mathbb{E}[\fcc]=\mathbb{E}\left[\sum_{i=1}^{N}{\mathds{1}\left[CT\left(c_i\right)\geq {CT}_{\rob}(c_i)\right]}\right]
\end{dmath*}
Using the expectation rules, we can insert the expectation sign inside the summation: \begin{dmath*}
\mathbb{E}[\fcc]=\sum_{i=1}^{N}{\mathbb{E}\left[\mathds{1}\left[{CT}_{\opp}(c_i)\geq {CT}_{\rob}(c_i)\right]\right]}
\end{dmath*}

So now, let us prove that $\mathbb{E}\left[\mathds{1}\left[{CT}_{\opp}(c_i)\geq {CT}_{\rob}(c_i)\right]\right]=\frac{1}{2}$. 
Using Markov's inequality, we know that: 
\begin{dmath*}[compact]
\mathbb{E}\left[\mathds{1}\left[{CT}_{\opp}(c_i)\geq {CT}_{\rob}(c_i)\right]\right]={P\left({CT}_{\opp}(c_i)\geq {CT}_{\rob}(c_i)\right)}
\end{dmath*}
So we need to prove that the probability for the covering-time of some cell $c_i$ by $S_{\opp}$ being greater than some constant value ${CT}_{\rob}(c_i)$, where ${CT}_{\rob}(c_i)$ is uniform in range $[0,1]$, equals to $\frac{1}{2}$.
Indeed, this is true, and here is why: both ${CT}_{\opp}(c_i)$ and ${CT}_{\rob}(c_i)$ are considered as i.i.d variables, each one of them is uniformly distributed between $0$ and $N$. Then, the probability that one is greater than the other is $\frac{1}{2}$, as can be deduced from the figure \ref{figures: covering-times}.

\begin{figure}[tb]
    \centering
    \begin{tikzpicture}
        \draw[very thin, gray!30, step=0.5 cm](-0.5,-0.5) grid (2.5,2.5);
    
        \fill [gray!60, domain=0:2, variable=\x]
          (-1, 0)
          -- plot ({\x}, {\x})
          -- (2, 0)
          -- cycle;
    
        \draw [thick] [->] (-0.5,0)--(2.5,0) node[right, below] {$CT_{\opp}(c_i)$};
    
        \draw [thick] [->] (0,-0.5)--(0,2.5) node[above, left] {$CT_{\rob}(c_i)$};
        \draw [domain=0:2, variable=\x]
          plot ({\x}, {\x}) node[right] at (0.5,2) {$CT_{\opp}(c_i)=CT_{\rob}(c_i)$};

    \end{tikzpicture}
    \caption{${CT}_{\opp}(c_i) > {CT}_{\rob}(c_i)$}
    \label{figures: covering-times}
   \end{figure}

Using the above, we get that:
\begin{dmath*}[compact]
\mathbb{E}[\fcc]=\sum_{i=1}^{N}{\mathbb{E}\left[\mathds{1}\left[CT\left(c_i\right)\geq i\right]\right]}={\sum_{i=1}^{N}{\frac{1}{2}}=\frac{N+1}{2}}
\end{dmath*}
\end{proof}

\subsection{Partial Information - $i_\opp$ is unknown, $S_\opp$ is known - asymmetrical-Knowledge} 


In this case, where \rob knows $S_\opp$, but not $i_\opp$, we examine whether \rob can achieve anything better than playing \cros, given that is is given more information: According to Theorem \ref{theorems: 2d max fcc unknown io} ,it can not. The best $\mathbb{E}[\fcc]$ \rob can achieve is $\frac{N+1}{2}$.

We present a new strategy, \coos(\ref{coos algorithm}), where $S_\rob$ is set to be the opponents strategy, $S_\opp$. 
\begin{algorithm}
\begin{algorithmic}
	\STATE Choose $S_\rob \leftarrow S_\opp$
    \STATE $c_i \leftarrow i_\rob$
    \LOOP
    	\STATE $c_i \leftarrow S_\rob(c_i)$
    \ENDLOOP
  
\end{algorithmic}
\caption{\coos\label{coos algorithm}}
\end{algorithm}

We prove in Theorem \ref{theorems: coos stupid}, sadly, that this does not increase the expected \fcc, as we hoped for.
\begin{theorem} \label{theorems: coos stupid}
When $IM_\rob=\lbrace S_\opp , i_\emptyset \rbrace$, then \[\mathbb{E}[\fcc_\rob(\coos, S_\opp)]=\frac{N+1}{2}\]
\end{theorem}

We then prove the more important Theorem \ref{theorems: 2d max fcc unknown io}, which brings a surprising result: the knowledge about $S_\opp$ is irrelevant to \opp, and can not help him achieve anything better than random-like results.

\begin{theorem}\label{theorems: 2d max fcc unknown io}
When $IM_\rob=\lbrace S_\opp , i_\emptyset \rbrace$, then 
\begin{dmath*}[compact]
\max \lbrace \mathbb{E}[\fcc_\rob(S_\rob, S_\opp)]\rbrace=\mathbb{E}[\fcc_\rob(S_\rob, S_\opp)]=\frac{N+1}{2}
\end{dmath*}
\end{theorem}

\begin{proof}
First of all, recall our previous definition of covering time: the Covering-Time of cell $c_i$ by some robot, using the coverage strategy $S_{\rob}$ is the time the robot first appears in $c_i$ according to $S_{\rob}$. This value is denoted by ${CT}_{S_{\rob}}(c_i)$. 
Now, since $S_{\rob}$ and $S_{\opp}$ are optimal-cyclic-coverage strategies, they start and stop at the same position, visiting each cell only once (since we assume a rectangular world with no obstacles, such a path exists). We can say that each strategy create a 'chain' from all the cells in the world $c_0,...,c_{N-1}$. In this chain, the relative place a cell $c_i$ appears in is its ${CT}_{S_{\rob}}(c_i)$ (We are talking about \rob,  from now on, but all of this is the same for \opp). Here is an illustration of our covering-chains:
\begin{figure}[tb]
\centering
\subfloat[An example for a covering chain, where coverage path starts at $c_0$]{
\begin{tikzpicture}[->,>=stealth',shorten >=1pt,auto,node distance=1.5cm,
                    semithick]
  \tikzstyle{every state}=[fill=red,draw=none,text=white]

  \node[initial below,state, label={CT=0}] (A)              {$c_0$};
  \node[state, label={CT=1}]         (B) [right of=A] {$c_1$};
  \node[state, label={CT=2}]         (C) [right of=B] {$c_2$};
  \node[state, label={CT=3}]         (D) [right of=C] {$c_3$};
  \node[state, label={CT=4}]         (E) [right of=D] {$c_4$};

  \path (A) edge              node {} (B)
        (B) edge              node {} (C)
        (C) edge              node {} (D)
        (D) edge              node {} (E)
        (E) edge [bend left] node {} (A);
\end{tikzpicture}
% \caption{An example for a covering chain, where coverage path starts at $c_0$}
}
\newline
\subfloat[An example for a covering chain, where coverage path starts at $c_2$.]{
\begin{tikzpicture}[->,>=stealth',shorten >=1pt,auto,node distance=1.5cm,
                    semithick]
  \tikzstyle{every state}=[fill=red,draw=none,text=white]

  \node[state, label={CT=3}] (A)              {$c_0$};
  \node[state, label={CT=4}]         (B) [right of=A] {$c_1$};
  \node[initial below,state, label={CT=0}]         (C) [right of=B] {$c_2$};
  \node[state, label={CT=1}]         (D) [right of=C] {$c_3$};
  \node[state, label={CT=2}]         (E) [right of=D] {$c_4$};

  \path (A) edge              node {} (B)
        (B) edge              node {} (C)
        (C) edge              node {} (D)
        (D) edge              node {} (E)
        (E) edge [bend left] node {} (A);
\end{tikzpicture}

% \caption{An example for a covering chain, where coverage path starts at $c_2$.}
}
\end{figure}


Now, understand this: each starting position $i_r$ determines the covering time of all the cells $c_0,...,c_{N-1}$; Since we assumed the strategy is known before hand, then, for \opp,  the covering time is set after $i_\opp$ is known, and changing it changes for all the cells their covering time. Here is an illustration:

% \begin{figure}[h]
% \centering

% \end{figure}
As can be seen in the examples above, changing $i_\opp$ directly changes the CT values of all vertices accordingly.
As the reader can see, since $S_{\opp}$ is set, then for any $c_i$ we'll get $CT_{S_{\opp}}(c_i)=0$ if $i_\opp=c_i$, and $CT_{S_{\opp}}(c_i)=N-1$ if $i_0=c_{N-1}$. More generally: if $i_\opp=c_j$ where $j$ ranges from $0$ to $N-1$, then $CT_{S_{\opp}}(c_i)=\abs{i-j}\in [0,N-1]$.

We now move to the next part: converting the way we look on \fcc. As seen before, one can write the \fcc of a fixed problem (with all its variables known) as $\fcc(W,S_{\rob},S_{\opp},i_r,i_\opp)=\# \lbrace CT_{S_{\rob}}(c_i) \le CT_{S_{\opp}}(c_i)\rbrace \textsl{ where } c_i\in W$.
If before we wanted to find the expected \fcc expression: 
\begin{dmath*}
\mathbb{E}\left[\fcc\left(W, S_{\rob}, i_r, S_{\opp}, i_\opp\right) \mid W, S_{\rob}, i_r, S_{\opp}\right]=
\frac{1}{N}\sum_{i_\opp\in W}{\fcc\left(W,S_{\rob},i_r,S_{\opp},i_\opp\right)}
\end{dmath*}
We now consider the following expression instead:
\begin{dmath*}
\mathbb{E}\left[\fcc\left(W, S_{\rob}, i_r, S_{\opp}, i_\opp\right) \mid W, S_{\rob}, i_r, S_{\opp}\right]=
\frac{1}{N}\sum_{i_\opp\in W}{\sum_{c_i\in W}{\mathds{1}\left[CT_{S_{\rob},i_r}(c_i) \le CT_{S_{\opp},i_\opp}(c_i)\right]}}
\end{dmath*}

If we change the order of summation, we can use what we know about ranging over the initial position and get:
\begin{dmath*}
\mathbb{E}\left[\fcc\left(W, S_{\rob}, i_r, S_{\opp}, i_\opp\right) \mid W, S_{\rob}, i_r, S_{\opp}\right]=
\frac{1}{N}\sum_{i_\opp\in W}{\sum_{c_i\in W}{\mathds{1}\left[CT_{S_{\rob},i_r}(c_i) \le CT_{S_{\opp},i_\opp}(c_i)\right]}}=
\frac{1}{N}\sum_{c_i\in W}{\sum_{i_\opp\in W}{\mathds{1}\left[CT_{S_{\rob},i_r}(c_i) \le CT_{S_{\opp},i_\opp}(c_i)\right]}}=
\frac{1}{N}\sum_{c_i\in W}{\# \lbrace CT_{S_{\rob},i_r}(c_i) \le CT_{S_{\opp},i_\opp}(c_i)\rbrace}=
\frac{1}{N}\sum_{c_i\in W}{N-CT_{S_{\rob},i_r}(c_i)}=
\frac{1}{N}(1+2+\ldots+N)=\frac{N+1}{2}
\end{dmath*}
Where the before-last equation comes from what we said we before: the number of times that $CT_{S_{\opp},i_\opp}(c_i)$ is greater or equal to $CT_{S_{\rob},i_r}(c_i)$ (can be considered as some constant $C$), when ranging over all cells as $i_\opp$, ranges itself from $N$ (if $i_\opp$ is exactly one cell after $c_i$) to $1$ (if $i_\opp$ is exactly one cell before $c_i$).
\end{proof}


\subsection{Partial Information - $S_\opp$ is unknown, $i_\opp$ is known - asymmetrical Knowledge} 
This is the last information model we want in the asymmetrical knowledge case. In this case, we have yet been unable to prove superiority of any better-than-random strategy.
Checking our simulations, we know that indeed, there are some strategies that yield better results; That is, there are some $S^{better}_{\opp}\in \mathcal{S}$ that yield $\mathbb{E}[\fcc(S_\opp=S^{better}_{\opp})] > \frac{N+1}{2}$, but we did not yet manage to prove what character made them better than other. So we know such better strategies exists.

We did prove, in Theorem \ref{theorems: 2d known io unknown so} is that playing the same as your opponent is irrelevant.
\begin{theorem} \label{theorems: 2d known io unknown so}
When $IM_\rob = \lbrace S_\emptyset, i_o \in \w \rbrace$, then \begin{dmath*}[compact]
\mathbb{E}{[\fcc_\rob(S_\rob=S_\opp, S_\opp)]} ={\mathbb{E}[\fcc_\rob(S_\rob=\cros, S_\opp)]} = \frac{N+1}{2}
\end{dmath*}
\end{theorem}

This part is remained to be investigated. 

\section{CONCLUSIONS}

A conclusion section is not required. Although a conclusion may review the main points of the paper, do not replicate the abstract as the conclusion. A conclusion might elaborate on the importance of the work or suggest applications and extensions. 

% \addtolength{\textheight}{-12cm}   % This command serves to balance the column lengths
                                  % on the last page of the document manually. It shortens
                                  % the textheight of the last page by a suitable amount.
                                  % This command does not take effect until the next page
                                  % so it should come on the page before the last. Make
                                  % sure that you do not shorten the textheight too much.

%%%%%%%%%%%%%%%%%%%%%%%%%%%%%%%%%%%%%%%%%%%%%%%%%%%%%%%%%%%%%%%%%%%%%%%%%%%%%%%%



%%%%%%%%%%%%%%%%%%%%%%%%%%%%%%%%%%%%%%%%%%%%%%%%%%%%%%%%%%%%%%%%%%%%%%%%%%%%%%%%



%%%%%%%%%%%%%%%%%%%%%%%%%%%%%%%%%%%%%%%%%%%%%%%%%%%%%%%%%%%%%%%%%%%%%%%%%%%%%%%%
\section*{APPENDIX}

Appendixes should appear before the acknowledgment.

\section*{ACKNOWLEDGMENT}

The preferred spelling of the word �acknowledgment� in America is without an �e� after the �g�. Avoid the stilted expression, �One of us (R. B. G.) thanks . . .�  Instead, try �R. B. G. thanks�. Put sponsor acknowledgments in the unnumbered footnote on the first page.



%%%%%%%%%%%%%%%%%%%%%%%%%%%%%%%%%%%%%%%%%%%%%%%%%%%%%%%%%%%%%%%%%%%%%%%%%%%%%%%%

\bibliographystyle{abbrv}
\bibliography{SharedParts/refs}

\end{document}
