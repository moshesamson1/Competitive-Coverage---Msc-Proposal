Let \rob and \opp be two robots operating in an obstacle-free grid \w of size $N=m \times n$. Both robots move in the four basic directions (North, South, East, West). Consider \rob to be our robot-of-interest, and \opp to be the opponent. 
The goal of each robot is to cover the area, that is, find a path (denoted as the coverage path) that visits  each point in the area at least once. We define a coverage strategy of a robot as the coverage path, including the order of cells visited (specifically in a cyclic coverage path, the strategy indicates both the cells' ordering, and the direction of movement---clockwise or counterclockwise), or a behavior (for example: opponent chooses randomly, at each step, one of its four neighbors, and go there). We denote \rob's and \opp's strategies by $S_\rob,S_\opp\in \mathcal{S}$, respectively, where $\cal{S}$ stands for the possible strategies space. In the offline version of the competitive coverage problem, $S_\rob$ and $S_\opp$ are computed in advance (before the execution), and mostly consist of cells permutation, where in the online problem, the strategies are behaviors, able to adjust online to the environment.

\opp is covering \w using an optimal coverage strategy, that is, it follows a path guaranteeing coverage in minimal time (we based our experiments on the Spanning-Tree Coverage algorithm \cite{gabriely2001spanning}). \rob's goal is to cover as many cells as possible from \w {\em before they are visited by \rob}. The calculation of the covering strategy of \rob, $S_\rob$, is based on its initial location, $i_r$. The initial position of \opp, $i_\opp$, is not necessarily known to \rob.

The way we define the problem, \rob can be given $i_\opp$, $S_\opp$, both or neither. These types of information are called {\em Information Models}, and defined as follows:
\begin{definition}[\textbf{Information Model}]
Information Model $IM \in \lbrace \varnothing, \lbrace S,I\rbrace \rbrace$, represents the knowledge a robot has on its opponent, where $S\in \mathcal{S} \cup \lbrace S_\emptyset \rbrace$, where $S_\emptyset$ stands for an unknown strategy, and $I \in W \cup \lbrace \w_\emptyset \rbrace$, where $\w_{\emptyset}$ refers to an unknown initial point.
If $IM=\varnothing$ then the player of interest does not know its opponent exists.
Let $IM_\rob$ be the information model \rob is given about \opp, and let $IM_\opp$ be the information model \opp is given about \rob.
\end{definition}
\rob thrives first to maximize the number of cells it covers before \opp. Denote this value by 
\begin{dmath*}[compact]
\fcc_{x}= 
\# \lbrace \textnormal{ Cells First Covered by } x \in \lbrace\rob,\opp\rbrace \rbrace 
\end{dmath*}
When deciding between options with the same \fcc value, \rob will choose the one that yields the fastest world's coverage time - the time it takes \rob to covers all of \w.

\opp can operate with or without the knowledge of \rob's existence. We refer to these cases as the {\em symmetric and asymmetric competitive coverage}, respectively.
In the {\em asymmetric} case, $IM_\opp=\varnothing$, and in the {\em symmetric} case, $IM_\opp \neq \varnothing$. In both, $IM_\rob \neq \varnothing $.

Considering all we said above, we define the competitive coverage problem:
\begin{mdframed}[backgroundcolor=gray!20] 
\begin{definition}[\textbf{Competitive Coverage Problem}]
Let \w be a finite, obstacles-free grid of size $N$. Given $IM_\opp=\lbrace S_\opp ,i_\opp \rbrace$ to \rob and $IM_\rob \in \lbrace \varnothing, \lbrace S_\rob, i_\rob \rbrace \rbrace$ to \opp, find $S_\rob^\star \in \mathcal{S}$ s.t.
\begin{dmath*}[compact]
S_\rob^\star =\argmax_{S_\rob\in\mathcal{S}} \lbrace \fcc_{r}(S_\rob, S_\opp, i_\rob, i_\opp ) \rbrace
\end{dmath*}
\end{definition}
\end{mdframed}
