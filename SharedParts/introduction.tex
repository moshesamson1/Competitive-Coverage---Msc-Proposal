The robotic coverage problem is one of the fundamental problems in robotic research, and as such has received considerable attention in the past two decades \cite{galceran2013survey}. The problem has its theoretical interest, but is of special interest due to its immediate applicability in real world settings, such as cleaning, coating, demining and search and rescue. 

In the original problem of robotic coverage, a robot's goal is to determine a path that will visit each point in a given area at least once, usually while minimizing the time for completion \cite{galceran2013survey}. The problem has been examined in different settings, for example offline vs. online coverage \cite{gabriely2001spanning,agmon2008giving}, where the environment map is given in advance or discovered during the execution (respectively), coverage with the presence of threats that might stop the robot \cite{yehoshua2014safest}, and continuous vs. discrete domains \cite{gabriely2001spanning,yang2004neural}. In the multi-robot coverage problem, the coverage is a collaborative effort: each point in the area should be visited at least once by some robot from the team, and the common goal is to minimize the maximal working time of some robot from the team. 

In this work we formally define a new variant of the coverage problem, {\em competitive coverage}, in which robots do not work collaboratively, but competitively. More formally, two robots, \rob and \opp, are to cover a given area represented as a grid, and our goal is to maximize the number of cells \rob covers first, before they are covered by \opp.

The problem can be classified as {\em symmetric} or {\em asymmetric}, which refers to wither \opp even knows about \rob's existence or not, regardless to what is the extent of such information. We first examine the {\em asymmetric} variant of the problem, in which \opp operates without the knowledge of \rob's existence, and \rob knows it should compete with \opp. The problem is also modeled by the level of information one robot has on the other (beside its existence): initial location and strategy/coverage-path. We consider four different models: (i) \rob knows the initial location of \opp and its planned coverage path ; (ii) \rob knows only the path, but does not know \opp's initial location ; (iii) \rob knows \opp's initial location, but not its coverage path ; (iv) \rob does not know both \opp's path and its location.

Solving the competitive coverage problem, in some cases, is as computationally hard as solving the original coverage problem \cite{arkin2000approximation}. For the sake of the analysis, we consider environments in which an optimal coverage path can be computed in polynomial time, for example by using the Spanning-Tree Coverage Algorithm \cite{gabriely2001spanning}, which generate, cyclic coverage paths under some assumptions on the environment. 

We therefore present an optimal algorithm for \opp in the full-information case, and show that, surprisingly, having some information is equivalent (in the average case) to having no information at all. That is, if \rob has information only on \opp's location {\em or} on its path, its optimal behavior is exactly as if it does not know any of those. We support our theoretical analysis with computer simulation, demonstrating our findings. 