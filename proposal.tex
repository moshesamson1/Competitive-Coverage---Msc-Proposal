\documentclass[a4paper,english,10pt]{article}


%Using Lines
\usepackage{algorithm}
\usepackage{algorithmic}
\usepackage{amsmath}
\usepackage{relsize}
\usepackage{amsfonts}
\usepackage{amssymb}
\usepackage{mathtools}
\usepackage{graphicx}
\usepackage{epsfig}
\usepackage{color}
\usepackage{mathtools}
\usepackage{tikz}
\usepackage{relsize}
\usepackage{float}
\usepackage{dsfont}
\usepackage{hyperref}
\usepackage[nameinlink]{cleveref}
\newcommand{\bigqm}[1][1]{\text{\larger[#1]{\textbf{?}}}}
\usepackage{tikz,fullpage}
\usetikzlibrary{arrows, petri, topaths}
\usepackage{tkz-berge}
\usepackage[position=top]{subfig}
\usepackage{verbatim}
\usepackage{amsthm}
\usepackage{pgf}
\usepackage{tikz}
\usepackage{fancyhdr, setspace, color, soul}
\usepackage{geometry,polyglossia,fontspec,csquotes, doi}
\usepackage{ucs}
\usepackage[utf8x]{inputenc}
% \usepackage[english,hebrew]{babel}
\usepackage{breqn}
\usepackage{mdframed}
\usepackage{dsfont}
\usepackage{tabularx}
\usepackage{xcolor}
\usepackage{xspace}
\usepackage{cjhebrew}
\usepackage{caption}
\usepackage{subcaption}
\usepackage{subfig}
\usepackage{ascii}
\usepackage{multirow}
\usepackage{xfrac}

\usetikzlibrary{arrows, automata, backgrounds,snakes}
% End

%language

% \usepackage{polyglossia}
% \usepackage{fancyhdr}
% \pagestyle{fancy}
% \lhead{H.W. \#1}
% \usepackage{bidi}
% \usepackage[utf8]{inputenc}
% \usepackage[T1]{fontenc}
% \usepackage[iso,english]{isodate}
% \newfontfamily {\H}[Scale=1]{David CLM}
% \newfontfamily {\Hb}[Scale=1]{David CLM Bold}
\newfontfamily\hebrewfont{Times New Roman}[Script=Hebrew]

% define MACROS
\newcommand{\len}{15}
\newcommand{\LPart}{0.4}
\DeclarePairedDelimiter{\ceil}{\lceil}{\rceil}
\DeclarePairedDelimiter{\floor}{\lfloor}{\rfloor}
\newtheorem{theorem}{Theorem}[section]
\newtheorem{corollary}{Corollary}[theorem]
\newtheorem{lemma}[theorem]{Lemma}
\newtheorem*{remark}{Remark}
\newcounter{casenum}
\newenvironment{caseof}{\setcounter{casenum}{1}}{\vskip.5\baselineskip}
\newcommand{\case}[2]{\vskip.5\baselineskip\par\noindent {\bfseries Case \arabic{casenum}:} #1\\#2\addtocounter{casenum}{1}}
\newcommand\rob{\ensuremath{r}\xspace}
\newcommand\opp{\ensuremath{o}\xspace}
\newcommand{\w}{\ensuremath{W}\xspace}
\newcommand{\fcc}{\ensuremath{FCC}\xspace}
\newcommand{\gipc}{\ensuremath{GIPC}\xspace}
\newcommand{\cros}{\ensuremath{CROS}\xspace}
\newcommand{\coos}{\ensuremath{COS}\xspace}
\newcommand{\gn}{\ensuremath{GN}\xspace}
\newcommand{\gf}{\ensuremath{GF}\xspace}
\newcommand{\go}{\ensuremath{GO}\xspace}
\DeclarePairedDelimiter\abs{\lvert}{\rvert}%
\DeclareMathOperator*{\argmax}{arg\,max} % Jan Hlavacek
\allowdisplaybreaks[2]
\newtheorem{definition}{Definition}
\def\checkmark{\tikz\fill[scale=0.4](0,.35) -- (.25,0) -- (1,.7) -- (.25,.15) -- cycle;}
\def\uncheckmark{$\mathbin{\tikz [x=1.4ex,y=1.4ex,line width=.2ex] \draw (0,0) -- (1,1) (0,1) -- (1,0);}$}
\newcommand{\Cross}{$\mathbin{\tikz [x=1.4ex,y=1.4ex,line width=.2ex, red] \draw (0,0) -- (1,1) (0,1) -- (1,0);}$}%

\makeatletter
\def\BState{\State\hskip-\ALG@thistlm}
\makeatother
% end


%%%%%% Latex Configuration
\renewcommand{\topfraction}{0.85}
\renewcommand{\textfraction}{0.1}
\renewcommand{\algorithmiccomment}[1]{// #1}
%%%%%
\long\def\symbolfootnote[#1]#2{\begingroup%
\def\thefootnote{\fnsymbol{footnote}}\footnote[#1]{#2}\endgroup}


%\setotherlanguage[numerals=western]{hebrew}
%\setdefaultlanguage{english}
\setmainlanguage{english}

\begin{document}
% \setdefaultlanguage{english}
% \setotherlanguage{hebrew}

\begin{titlepage}
\begin{center}


\includegraphics[width=0.9\textwidth]{Images/logo.jpg}\\[1cm]
\textsc{\LARGE Bar Ilan University}\\[1.5cm]
\textsc{\Large M.Sc Proposal}\\[0.5cm]
%\hrule \\[0.4cm]
%{\huge \bfseries Speeding Frontier-Based Exploration by
%Using Semantic Labeling}\\[0.4cm]
%\hrule \\[1.5cm]

\hrule
{ \vspace{2 mm} }
{ \huge \bfseries Competitive Coverage}
{ \vspace{3 mm} }
\hrule
{ \vspace{8 mm} }

\hrule
{ \vspace{2 mm} }

{ \huge \bfseries  \huge \bfseries \<b`yyt hkyswy ht.hrwty> }
{ \vspace{3 mm} }
\hrule
{ \vspace{8 mm} }

% \author{Moshe N. Samson\\
% advised by Noa Agmon\\
% The MAVERICK Group, Computer Science Department\\
% Bar Ilan University\\
% Ramat Gan, Israel 52900\\
% \tt\small samson.moshe@gmail.com

%author and supervisor
\begin{minipage}{0.4\textwidth}
\begin{flushleft} \large
\emph{Author:}\\
Moshe N. Samson 

\end{flushleft}
\end{minipage}
\begin{minipage}{0.4\textwidth}
\begin{flushleft} \large
\emph{Supervisor:} \\
Noa Agmon
\end{flushleft}
\end{minipage}


\vfill


\large{The SMART Group, Computer Science Department\\
Bar Ilan University\\
Ramat Gan, Israel 52900\\
\tt\small samson.moshe@gmail.com}

\vfill

\today

\end{center}
\end{titlepage}

%opening
\title{Competitive Coverage}
\author{Moshe N. Samson\\
advised by Noa Agmon\\
The SMART Group, Computer Science Department\\
Bar Ilan University\\
Ramat Gan, Israel 52900\\
\tt\small samson.moshe@gmail.com
}

\tableofcontents
\maketitle

\section{Introduction}
The robotic coverage problem is one of the fundamental problems in robotic research, and as such has received considerable attention in the past two decades \cite{galceran2013survey}. The problem has its theoretical interest, but is of special interest due to its immediate applicability in real world settings, such as cleaning, coating, demining and search and rescue. 

In the original problem of robotic coverage, a robot's goal is to determine a path that will visit each point in a given area at least once, usually while minimizing the time for completion \cite{galceran2013survey}. The problem has been examined in different settings, for example offline vs. online coverage \cite{gabriely2001spanning,agmon2008giving}, where the environment map is given in advance or discovered during the execution (respectively), coverage with the presence of threats that might stop the robot \cite{yehoshua2014safest}, and continuous vs. discrete domains \cite{gabriely2001spanning,yang2004neural}. In the multi-robot coverage problem, the coverage is a collaborative effort: each point in the area should be visited at least once by some robot from the team, and the common goal is to minimize the maximal working time of some robot from the team. 

In this work we formally define a new variant of the coverage problem, {\em competitive coverage}, in which robots do not work collaboratively, but competitively. More formally, two robots, \rob and \opp, are to cover a given area represented as a grid, and our goal is to maximize the number of cells \rob covers first, before they are covered by \opp.

The problem can be classified as {\em symmetric} or {\em asymmetric}, which refers to wither \opp even knows about \rob's existence or not, regardless to what is the extent of such information. We first examine the {\em asymmetric} variant of the problem, in which \opp operates without the knowledge of \rob's existence, and \rob knows it should compete with \opp. The problem is also modeled by the level of information one robot has on the other (beside its existence): initial location and strategy/coverage-path. We consider four different models: (i) \rob knows the initial location of \opp and its planned coverage path ; (ii) \rob knows only the path, but does not know \opp's initial location ; (iii) \rob knows \opp's initial location, but not its coverage path ; (iv) \rob does not know both \opp's path and its location.

Solving the competitive coverage problem, in some cases, is as computationally hard as solving the original coverage problem \cite{arkin2000approximation}. For the sake of the analysis, we consider environments in which an optimal coverage path can be computed in polynomial time, for example by using the Spanning-Tree Coverage Algorithm \cite{gabriely2001spanning}, which generate, cyclic coverage paths under some assumptions on the environment. 

We therefore present an optimal algorithm for \opp in the full-information case, and show that, surprisingly, having some information is equivalent (in the average case) to having no information at all. That is, if \rob has information only on \opp's location {\em or} on its path, its optimal behavior is exactly as if it does not know any of those. We support our theoretical analysis with computer simulation, demonstrating our findings. 

\section{Background and Related Work}
\input{SharedParts/related_work.tex}

\section{Competitive Coverage: Definition}
Let \rob and \opp be two robots, operating in an obstacle-free grid \w of size $N=m \times n$. Both robots move in the four basic directions (North, South, East, West). Consider \rob to be our robot-of-interest, and \opp to be the opponent. 
The goal of each robot is to cover the area, that is, find a path (denoted as the coverage path) that visits  each point in the area at least once. We define a coverage strategy of a robot as the coverage path, including the order of cells visited (specifically in a cyclic coverage path, the strategy indicates both the cells' ordering, and the direction of movement---clockwise or counterclockwise), or a behavior (for example: opponent chooses randomly, at each step, one of its four neighbors, and go there). We denote \rob's and \opp's strategy by $S_\rob,S_\opp\in \mathcal{S}$ (respectively), where $\cal{S}$ stands for the possible strategies space. I the offline version of the competitive coverage problem, $S_\rob$ and $S_\opp$ are computed in advance (before the execution), and mostly consists of cells permutation, where in the online problem, the strategies are behaviors, able to adjust online to the environment.

\opp is covering \w using an optimal coverage strategy, that is, it follows a path guaranteeing coverage in minimal time (we use in our experiment the Spanning-Tree Coverage (STC) algorithm \cite{gabriely2001spanning}). \rob's goal is to cover as many cells as possible from \w {\em before they are visited by \rob}. The calculation of the covering strategy of \rob, $S_\rob$, is based on its initial location, $i_r$. The initial position of \opp, $i_\opp$, is not necessarily known to \rob.

The way we define the problem, \rob can be given $i_\opp$, $S_\opp$, both or neither. These types of information are called {\em Information Models}, and defined as follows:
\begin{definition}[\textbf{Information Model}]
Information Model $IM \in \lbrace \varnothing, \lbrace S,I\rbrace \rbrace$, represents the knowledge a robot has on its opponent, where $S\in \mathcal{S} \cup \lbrace S_\emptyset \rbrace$, where $S_\emptyset$ stands for an unknown strategy, and $I \in W \cup \lbrace \w_\emptyset \rbrace$, where $\w_{\emptyset}$ refers to an unknown initial point. 
If $IM=\varnothing$ then the player of interest does not know its opponent exists.
Let $IM_\rob$ be the information model \rob is given about \opp, and let $IM_\opp$ be the information model \opp is given about \rob.
\end{definition}

\opp can operate with or without the knowledge of \rob's existence. We refer to these cases as the {\em symmetric and asymmetric competitive coverage}.
In the {\em asymmetric} case, $IM_\opp=\varnothing$, and in the {\em symmetric} case, $IM_\opp \neq \varnothing$. In both cases, $IM_\rob \neq \varnothing $.

\rob thrives first to maximize the number of cells it covers before \opp. Denote this value by 
\begin{dmath*}[compact]
\fcc_{x}(S_\rob, S_\opp, i_\rob, i_\opp )= 
\# \lbrace \textnormal{ First Covered Cells by } x \in \lbrace\rob,\opp\rbrace \rbrace 
\end{dmath*}
When deciding between options with the same \fcc value, \rob will choose the one that yields the fastest world's coverage time - the time it takes \rob to covers all of \w.

Considering all we said above, we define the competitive coverage problem:
\begin{mdframed}[backgroundcolor=gray!20] 
\begin{definition}[\textbf{Competitive Coverage Problem}]
Let \w be a finite, obstacles-free grid of size $N$. Let $IM_\rob=\lbrace S_\opp ,i_\opp \rbrace$ the information model given to \rob, and $IM_\opp \in \lbrace \varnothing, \lbrace S_\rob, i_\rob \rbrace \rbrace$. Find:
\begin{dmath*}[compact]
S_\rob^\star \in \mathcal{S} \textnormal{ s.t. } S_\rob^\star =\argmax_{S_\rob\in\mathcal{S}} \lbrace \fcc_{r}(S_\rob, S_\opp, i_\rob, i_\opp ) \rbrace
\end{dmath*}
\end{definition}
\end{mdframed}



\section{Initial Results}
In this section we present results from a simple, 1D world, and from a more realistic 2D world. In each one, we divide our work between the different information-models available, checking and comparing between symmetric and asymmetric knowledge.

A summary of the different information models considered, and some of the obtained results, considering 1D and 2D worlds, is described in table \ref{Tables: Initial Results}.
  
\begin{table}[h]
\caption{Summary of results, with respect to the information held by \rob.}
\label{table_example}
\begin{center}
\begin{tabular}{|c|c|cc|c|c|}
\hline 
\multirow{2}{*}{World}  & \multirow{2}{*}{Symmetry Type} & \multicolumn{ 2}{|C|}{$IM$}  & \multirow{2}{*}{$\mathbb{\fcc}$} & \multirow{2}{*}{Optimal strategy}\\
& & $S_x$ & $i_x$ & &\\
\hline
\hline
\multirow{ 8}{*}{1D} & \multirow{ 4}{*}{Asymmetric} & \checkmark & \checkmark & Complex & \go \\
\cline{3-6}
& & \uncheckmark & \checkmark & Complex & \gf \\
\cline{3-6}
& & \checkmark & \uncheckmark & Complex & \gf \\
\cline{3-6}
& & \uncheckmark & \uncheckmark & Complex & \gf \\
\cline{2-6}
& \multirow{ 4}{*}{Symmetric} & \checkmark & \checkmark & ? & ? \\
\cline{3-6}
& & \uncheckmark & \checkmark & ? & ? \\
\cline{3-6}
& & \checkmark & \uncheckmark & ? & ? \\
\cline{3-6}
& & \uncheckmark & \uncheckmark & ? & ? \\
\cline{1-6}
\multirow{ 8}{*}{2D} & \multirow{ 4}{*}{Asymmetric} & \checkmark & \checkmark & $m\cdot n-\frac{\left(m+1\right)\left(m-1\right)}{3m}-\frac{\left(n+1\right)\left(n-1\right)}{3n}$ & $GIPC$ \\
\cline{3-6}
& & \uncheckmark & \checkmark & $\sfrac{(N+1)}{2}$ & \cros / \coos \\
\cline{3-6}
& & \checkmark & \uncheckmark & ? & ?\\
\cline{3-6}
& & \uncheckmark & \uncheckmark & $\sfrac{(N+1)}{2}$ & \cros\\
\cline{2-6}
& \multirow{ 4}{*}{Symmetric} & \checkmark & \checkmark & ? & ? \\
\cline{3-6}
& & \uncheckmark & \checkmark & ? & ? \\
\cline{3-6}
& & \checkmark & \uncheckmark & ? & ?\\
\cline{3-6}
& & \uncheckmark & \uncheckmark & ? & ?\\
\hline
\end{tabular}
\end{center}
\label{Tables: Initial Results}
\end{table}


\subsection{1D World} \label{sections:1D intro}
In the 1D world, we consider \w to be a single line, where \rob and \opp move along it. This is a degenerated case, but still important in terms of laying the basis for the rest of our research.

As mentioned above in the problem definition, each player holds an information model, that can contains the opponents initial position, strategy, neither or both. Considering that, we divide our research into 4 different information models:
\begin{enumerate}
\item \textbf{Full information} - $i_\opp$ is known, $S_\opp$ is known
\item \textbf{Zero-Information} - $i_\opp$ is known, $S_\opp$ is unknown
\item \textbf{Partial Information} -  $i_\opp$ is unknown, $S_\opp$ is known
\item \textbf{Partial Information} - $i_\opp$ is unknown, $S_\opp$ is unknown
\end{enumerate}

For each of these problems, we would like to know what is the best strategy for \rob to play, maximizing $\fcc_\rob$.

Consider two main strategies:

$GoNear(\gn)$ - go towards the nearer edge first, until reaching it or meeting the opponent, then turn around and go all the way towards the other edge.

$GoFar(\gf)$ - go towards the far edge first, until reaching it or meeting the opponent, then turn around and go all the way towards the other edge.

And, for cases where its available and relevant, let us define:

$Go(to)Oppenet(\go)$ - go towards the opponent's initial position first, and when meeting, turn around and travel towards the edge until reaching it.

In the 1D world, we limited $S_\opp$ optional strategies to be either \gn or \gf, where \gn means going towards the closer edge first and then go back toward far edge, and \gf means going toward the far one first. 
Note that in a 1D world, a coverage algorithm usually requires that cells are revisited, thus we consider all complete strategies (with no reference to revisited cells).
% For \rob, we considered a third option, when optional, which is to go toward the opponent's initial position, and when meeting it, turning around and go back to the far edge.
% Note that in 1D world, there are no 'optimal' strategies (in the sense that every complete strategy must cover at least one cell twice), therefore, not like the 2D world, we do not enforce our strategies to be optimal, but only to be complete.

Assume the robots have, to some degree, some localization capabilities. This allows them to know what is the near edge and what is the far edge.
Therefore, we enumerate the 4 information models mentioned above, and for each one, we consider the different options for $S_\opp\in \lbrace \gn,\gf\rbrace$, and check what is the best response for $S_\rob$, and the \fcc it gains. We formulate the gained \fcc as a utility function, that depends on \opp's strategy and initial position.
For each such case, we either prove it mathematically, if possible, and demonstrate the results it using computer simulations of the gained \fcc from the utility function we derive for each such case.


\subsubsection{Information Models}
Let us show our initial results, regarding the four information models. In all cases, we assumed w.l.o.g that \rob is on the left half of \w.
\paragraph{Full Information - asymmetrical-Knowledge}
We started by proving the following lemmas:
\begin{lemma}
Assume that $ S_\opp=\gf$ and $i_\opp \in (0,i_\rob)$, then {\rob}s best response is $S_\rob=\gn$.
\end{lemma}
\begin{lemma}
Assume that $ S_\opp=\gn$ and $i_\opp \in (i_\rob, N/2)$, then {\rob}s best response is $S_\rob=\gf$.
\end{lemma}
\begin{lemma}
Assume that $ S_\opp=\gf$ and $i_\opp \in (N/2,N)$, then {\rob}s best response is $S_\rob=\gf$.
\end{lemma}
We used the above to prove theorem \ref{theorems: 1d full info asym So=go}:
\begin{theorem} \label{theorems: 1d full info asym So=go}
Assume $S_\opp=\go$, then \rob best response is $S_\rob=\go$. 
\end{theorem}

In a similar way, we deconstructed the other options for $S_\opp$, and proved theorems \ref{theorems: 1d full info asym So=gn} and \ref{theorems: 1d full info asym So=gf}:
\begin{theorem} \label{theorems: 1d full info asym So=gn}
Assume $S_\opp=\gn$, then \rob best response is $S_\rob=\go$.
\end{theorem}
\begin{theorem} \label{theorems: 1d full info asym So=gf}
Assume $S_\opp=\gf$, then \rob best response is $S_\rob=\go$.
\end{theorem}

We conclude and say that in all cases, {\rob}s best response is playing $S_\rob=\go$.

% % In this information model case, we started by claiming, and then proving mathematically, that for $S_\opp = \gf$, for all the different cases of $i_\opp$ (where \opp is to the left of \rob), \rob's best response is to play $\gn$.

% After proving the theory behind, we checked our work. For each and every one of the theorems and claims, we plotted 

% Then, we checked for $S_\opp = \gf$. We started by getting experimental results using simple simulations, then mathematically proved that \rob's best response is to go toward \opp initial position, and when meeting, turn around and go to the world's edge ahead.
% In figure \ref{figures:1D,partial,FullInfo,B case} we can see our simulations results, leading us to the conclusion we finally proved. (Explain graph).
% \begin{figure}[H]
% \includegraphics[width=\textwidth]{Images/GainedProfitp2SecondHalfWorld.png}
% \caption{Expected profits for different cases of $r$}
% \label{figures:1D,partial,FullInfo,B case}
% \end{figure}

% Note that in practice, this is the same response for $S_\opp = \gf$, so we can say that either way, \rob's best response is to go toward $i_\opp$, regardless to $S_\opp$.

\paragraph{Partial Information - $i_\opp$ is known, $S_\opp$ is unknown - asymmetrical-Knowledge}
This case was much simpler: according to the full knowledge case, for either $S_\opp \in \lbrace \gn,\gf \rbrace$ the best response is to go towards $i_\opp$, which in this case, is known.

This result is interesting: what we show is, actually, that knowing the opponent' strategy does not count for anything. This result stands in contrast to what we found in the 2D similar case, as will be shown.

\paragraph{Zero Information - asymmetrical-Knowledge}
In this case, we started by proving theorems \ref{theorems: 1D no info basic} and \ref{theorems: 1D not info not recognize}.
\begin{theorem} \label{theorems: 1D no info basic}
When $IM_\rob=lbrace S_\emptyset, i_\emptyset \rbrace$, \rob's best response is $S_\rob = \gf$.
\end{theorem}
\begin{theorem} \label{theorems: 1D not info not recognize}
When $IM_\rob=\varnothing$, \rob's best response is $S_\rob = \gn$.
\end{theorem}

Figures \ref{figures:1d unkown Io So=GN} and \ref{figures:1d unkown Io So=GF}, show that \rob's best response is $S_\rob=\gf$, but it will play it only when $IM_\rob \neq \varnothing$, otherwise \rob only wants to minimize its running time. This leads us to the conclusion described in \ref{theorems: utility of no information}.

\begin{theorem}\label{theorems: utility of no information}
$IM=\lbrace S_\emptyset, i_\emptyset \rbrace$ yields better results than $IM_\rob=\varnothing$.
\end{theorem}

Which means that in the zero-information, asymmetrical case, it is best to play $S_\rob=\gf$, and it is better than playing as if \opp is not exists. 

\paragraph{Partial Information - $i_\opp$ is unknown, $S_\opp$ is known - asymmetrical-Knowledge} \label{sections:1D unknown io known so}
In this case, we have yet been unsuccessful in proving the optimal strategy for \rob, maximizing its \fcc. We have constructed the expected \fcc functions for both $S_\opp = \gn$ and $S_\opp = \gf$ by checking each and different case for both options of available $S_\opp$s, and for all different ranges of optional $i_\opp$. An example for such plot is shown in figure \ref{figures:examples Sos utility}:

\begin{figure}
\begin{tikzpicture}[thick,framed]
%draw players
  \draw (0,0) -- (\len,0);
  \draw[snake=ticks,segment length=1cm] (0,0) -- (\len,0);
  \node[above] at (0,0) {$\textsc{0}$};% label the hinge
  \node[above] at (\len,0) {$\textsc{L+R}$};% label the hinge
  
  \filldraw[ball color=blue!80,shading=ball] (0.1*\len,0.5) circle
        (0.06cm) node[above]{$r$};% p1
  
  \filldraw[ball color=red!80,shading=ball] (0.8*\len,0.5) circle
        (0.06cm) node[above]{$o$};% p2
     
   % draw path
   \draw[->, blue] (0.1*\len,0.5) -- (0.0*\len,0.5)
	  	node[pos=0.7,above]{}; % path1 p2
   \draw[->, blue, dashed] (0.0*\len,0.4) -- (0.1*\len,0.4)
	  	node[pos=0.7,above]{}; % path1 p2
   \draw[-, green] (0.1*\len,0.4) -- (0.3*\len,0.4)
	  	node[pos=0.7,above]{}; % path1 p2
   \draw[->, blue] (0.3*\len,0.4) -- (0.55*\len,0.4)
	  	node[pos=0.7,above]{}; % path1 p2
   
   \draw[-, red, dashed] (0.8*\len,0.5) -- (\len,0.5)
	  	node[pos=0.5,above]{}; % path1 p1
   \draw[->, red, dashed] (\len,0.4) -- (0.55*\len,0.4)
	  	node[pos=0.5,above]{}; % path1 p1
   
   % draw axis
   \draw[<->] (0, -0.3) -- (0.45*\len, -0.3)
        node[pos=0.5,below]{$\textsc{L}$}; % Left hand
   \draw[<->] (0.45*\len, -0.3) -- (\len, -0.3)
        node[pos=0.5,below]{$\textsc{R}$}; % Right Hand
\end{tikzpicture}
\caption{An example for a single considered case. (Green is the extra part \rob is doing before \opp is reaching its initial position)}
\label{figures:examples Sos utility}
\end{figure}

Lemmas \ref{theorems: 1d unknown io so=GN},\ref{theorems: 1d unknown io so=GF} are yet left to be proved, and  their utility functions are displayed (for correctness-seeing purposes) in figures \ref{figures:1d unkown Io So=GN} and \ref{figures:1d unkown Io So=GF}.
\begin{theorem} \label{theorems: 1d unknown io so=GN}
$\mathbb{E}[\fcc_\rob (S_\rob=\gf, S_\opp=\gn)] > \mathbb{E}[\fcc_\rob (S_\rob=\gn, S_\opp=\gn)]$
\end{theorem}
\begin{theorem} \label{theorems: 1d unknown io so=GF}
$\mathbb{E}[\fcc_\rob (S_\rob=\gf, S_\opp=\gf)] > \mathbb{E}[\fcc_\rob (S_\rob=\gn, S_\opp=\gf)]$
\end{theorem}

% For both cases, it is clear that $S_\rob = \gf$ yield higher $\mathbb{E}[\fcc]$ than $S_\rob = \gn$, the functions are displayed in figure \ref{figures:1d unkown Io So=GN} and figure \ref{figures:1d unkown Io So=GF}.

\begin{figure}
    \centering
    \includegraphics[width=\textwidth]{Images/E1_E2_p2A_abs.png}
    \caption{$i_\rob$ values vs. $\mathbb{E}[\fcc_\rob]$, averaging gained $\fcc_\rob$  $0<i_\opp<100$, $S_\opp=\gn$ \gn is green and \gf is red}
    \label{figures:1d unkown Io So=GN}
\end{figure}

\begin{figure}
    \centering
    \includegraphics[width=\textwidth]{Images/E1_E2_w_wo_abs.png}
    \caption{$i_\rob$ values vs. $\mathbb{E}[\fcc_\rob]$, averaging gained $\fcc_\rob$  $0<i_\opp<100$, $S_\opp=\gf$ \gn is green and \gf is red}
    \label{figures:1d unkown Io So=GF}
\end{figure}

% For future work, we may consider our information model without localization capabilities, and thus different set of available strategies.

% \subsubsection{Symmetric Knowledge}
% Up to this point, we considered the asymmetric version of our problem, where \rob knows or does not know different things about \opp, and \opp acts as if it is alone, and it only goal is to cover as fast as possible.
% A more realistic model is the symmetric knowledge, where \opp knows as much about \rob as \rob knows about \opp.

% We will go though the four information models available, and will try to find the best strategy for \rob, taking in consideration $S_\opp$, which now will be more complex. Then, we will try to give numerical analyzing about the given \fcc. At the end, the same way as before, we check our results using simple and realistic simulations in order to verify our results.

\subsection{2D World}
Same as before, we started by dividing to different given information model: either \rob is given, or not, the initial location of \opp, $i_\opp$, and its strategy, $S_\opp$. Then, assuming a very basic behavior, where only \rob is aware of \opp's existence, and it must decide its strategy before the game begins, we tried to find the best strategy $S_\rob$ for \rob, given the information model.
Given this information model, we start by considering the asymmetric case in which only \rob is aware of \opp. Hence, we aim at determining the optimal strategy $S_\rob$ for \rob, based on the information models.

\subsubsection{Information Models}
Also here, we considered the four information models described in \ref{sections:1D intro}
% There are 4 basic information model: 
% \begin{enumerate}
% \item \textbf{Full information} - $i_\opp$ is known, $S_\opp$ is known
% \item \textbf{Zero-Information} - $i_\opp$ is known, $S_\opp$ is unknown
% \item \textbf{Partial Information} -  $i_\opp$ is unknown, $S_\opp$ is known
% \item \textbf{Partial Information} - $i_\opp$ is unknown, $S_\opp$ is unknown
% \end{enumerate}

\paragraph{Full Information - asymmetrical-Knowledge}

In the full information case we present a new algorithm, \gipc (\ref{algorithms: gipc}) that maximizes the \fcc target function.
The algorithm is as follows:
\begin{algorithm}
\begin{algorithmic}
	\STATE $i_p^{S_\rob S_\opp} \leftarrow $ Find Interception Point between $S_\rob$ and $S_\opp$
    \STATE GoTo $i_p^{S_\rob S_\opp}$, precede \opp by one step
    \LOOP
    	\IF {Cell $c_i$ \NOT Covered already}
        	\STATE GoTo $S_\opp(c_i)$
        \ENDIF
    \ENDLOOP
  
\end{algorithmic}
\caption{GIPC\label{algorithms: gipc}}
\end{algorithm}

\begin{theorem}
\gipc is correct, complete and optimal, and on a rectangular grid of size $m\times n$ it yields \[\mathbb{E}[\fcc] = m\cdot n-\frac{\left(m+1\right)\left(m-1\right)}{3m}-\frac{\left(n+1\right)\left(n-1\right)}{3n}\]
\end{theorem}

% Lastly, we do experiments. We start by running code simulations, to check that our algorithm is correct, that it is indeed the most preferable strategy, and that it does yield the computed expected \fcc.
% Then, we'll do realistic simulations using ROS, to see that $GIPC$ works correctly and as expected under real-world constraints, also.

\paragraph{Zero Information - asymmetrical-Knowledge}
In the zero-information case, \rob knows neither $i_\opp$ nor $S_\opp$. In fact, in this information model, \rob knows about \opp only that it exists.

We then introduce the \cros algorithm(\ref{algorithms: cros}), which may sound trivial, but we prove theoretically (theorem \ref{theorems: cros correctness and optimalty}) that it is the best strategy to play, and de-facto, knowing an opponents exists in the world, does not give \rob any advantage. Lastly, we proved the resulted $\mathbb{E}[\fcc]$ in \ref{cross efcc}.

\begin{algorithm}
\begin{algorithmic}
	\STATE Choose $S_\rob \in \mathcal{S}$ in random.
    \STATE $c_i \leftarrow i_\rob$
    \LOOP
    	\STATE $c_i \leftarrow S_\rob(c_i)$
    \ENDLOOP
  
\end{algorithmic}
\caption{\cros\label{algorithms: cros}}
\end{algorithm}

\begin{theorem} \label{theorems: cros correctness and optimalty}
\cros is correct, complete and when $IM_\rob=\lbrace S_\emptyset, i_\emptyset \rbrace$, is optimal.
\end{theorem}

\begin{theorem} \label{cross efcc}
Given $IM_\rob=\lbrace S_\emptyset, i_\emptyset \rbrace$, then $\mathbb{E}[\fcc_\rob(S_\rob=\cros)]=\frac{N+1}{2}$.
\end{theorem}

% Lastly, we do experiments. We perform simulations, showing our claims about the optimality of playing \cros (in this specific information model), and about the given expected \fcc it yields, are correct. 
% Lastly, we do hard experiments using ROS, to show that indeed, the expected \fcc when $S_\rob=\cros$ with no other information, and averaging over different strategies and initial positions, are indeed what we claim, and that it works under real-world constraints.

\paragraph{Partial Information - $i_\opp$ is unknown, $S_\opp$ is known - asymmetrical-Knowledge} 


In this case, what happens when \rob knows $S_\opp$, but not $i_\opp$? Can \rob achieve anything better than playing \cros, given that is is given more information? Can we set $S_\rob$ in a way that will yield better-than-random results, where $S_\rob$ can depend only on $S_\opp$? According to \ref{theorems: 2d max fcc unknown io} ,\rob cannot. The best $\mathbb{E}[\fcc]$ \rob can achieve is $\frac{N+1}{2}$.

Before proving \ref{theorems: 2d max fcc unknown io}, we wondered if by using the opponent strategy, \rob can achieve something. We present a new strategy, \coos(\ref{coos algorithm}), where $S_\rob$ is set to be the opponents strategy, $S_\opp$. 
\begin{algorithm}
\begin{algorithmic}
	\STATE Choose $S_\rob \leftarrow S_\opp$
    \STATE $c_i \leftarrow i_\rob$
    \LOOP
    	\STATE $c_i \leftarrow S_\rob(c_i)$
    \ENDLOOP
  
\end{algorithmic}
\caption{\coos\label{coos algorithm}}
\end{algorithm}

We then proved, sadly, that this does not give us anything (\ref{theorems: coos stupid}
\begin{theorem} \label{theorems: coos stupid}
When $IM_\rob=\lbrace S_\opp , i_\emptyset \rbrace$, then \[\mathbb{E}[\fcc_\rob(\coos, S_\opp)]=\frac{N+1}{2}\]
\end{theorem}

After that, we proved the more important theorem \ref{theorems: 2d max fcc unknown io}. This tells us a very surprising result: that the knowledge about $S_\opp$ is irrelevant to \opp, and can not help him achieve anything better than random-like results.

\begin{theorem}\label{theorems: 2d max fcc unknown io}
When $IM_\rob=\lbrace S_\opp , i_\emptyset \rbrace$, then \[\max \lbrace \mathbb{E}[\fcc_\rob(S_\rob, S_\opp)]\rbrace=\mathbb{E}[\fcc_\rob(S_\rob, S_\opp)]=\frac{N+1}{2}\]
\end{theorem}


\paragraph{Partial Information - $S_\opp$ is unknown, $i_\opp$ is known - asymmetrical Knowledge} 
This is the last information model we want in the asymmetrical knowledge case. In this case, we have yet been unable to prove priority of any better-than-random strategy.
Checking our simulations, we know that indeed, there are some strategies that yield better results; That is, there are some $S^{better}_{\opp}\in \mathcal{S}$ that yield $\mathbb{E}[\fcc(S_\opp=S^{better}_{\opp})] > \frac{N+1}{2}$, but we didn't manage yet to prove what character made them better than other. So we know such better strategies exists.

We did prove, in theorem \ref{theorems: 2d known io unknown so} is that playing the same as your opponent is irrelevant.
\begin{theorem} \label{theorems: 2d known io unknown so}
When $IM_\rob = \lbrace S_\emptyset, i_o \in \w \rbrace$, then \[\mathbb{E}[\fcc_\rob(S_\rob=S_\opp, S_\opp)] = \mathbb{E}[\fcc_\rob(S_\rob=\cros, S_\opp)] = \frac{N+1}{2}\]
\end{theorem}


% This is the last information model we want in the asymmetrical Knowledge case. Here, \rob gets the initial position of \opp, but not its strategy.
% As was shown before, the key component to improving the \fcc criterion is to know $i_\opp$. So, what do we know:
% Right now, we stand in a very interesting, even if a little frustrating: We checked and saw that \rob has better-than-other strategies to play when the initial position $i_\opp$ is known (and set) and averaging over different strategies. But we do not know what are the criteria for a good or bad strategy.

This part is remained to be investigated. 

\section{Work Plan}
The next steps in our work will be:
\begin{enumerate}
\item Analyze the symmetric game in 1D world.

Up to this point, we considered the asymmetric version of our problem, where \rob knows or does not know different things about \opp, and \opp acts as if it is alone, and it only goal is to cover as fast as possible.
A more realistic model is the symmetric knowledge, where \opp knows as much about \rob as \rob knows about \opp.

We will go though the four information models available, and will try to find the best strategy for \rob, taking in consideration $S_\opp$, which now will be more complex. Then, we will try to give numerical analyzing about the given \fcc. At the end, the same way as before, we check our results using simple and realistic simulations in order to verify our results.

\item Analyze the symmetric game in 2D world.
As mentioned above for 1D, we will do the same for 2D.
We'll start by considering the offline problem, where \rob and \opp need to choose, in advance, what should they do, considering that now \opp knows about \rob (either mere existence, initial position, strategy, or a combination of above).
Going through the four information models, we will try to find what is  best strategy $S_\rob$ for \rob. 
After that, we will check to see if when considering a finite number of strategies, we are to find some Nash-Equilibrium, where both players have the same information model, and they both prefer to use the same strategy, each of them prefer to stick with its strategy, considering the other one is doing the same. We'll then analyze the results using game-theory tools (for example: is the resulted N.E. is also pareto-optimal?).

\item Research the forth information model in 2D asymmetrical world.

As mentioned above, we saw that there are some strategies that yield better-than-random result. 
We would like to try and find some characteristics which indicates a strategy would be better than others.

\item Implements realistic simulation to 1D and 2D worlds for different cases of information models.

Finally, we would like and test our claims and theorems under real-world constraints. We will use ROS in order to implement the main algorithms and behaviors mentioned above, and check that they are indeed working correctly.
We would like to know what changes, if any, are needed in order for them to work correctly.

\end{enumerate}


\bibliographystyle{abbrv}
\bibliography{Share refs}

\end{document}
